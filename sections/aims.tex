The primary aim of this project is to develop a Universal \textbf{Language Server Protocol}\footnote{https://microsoft.github.io/language-server-protocol} (LSP) and \textbf{Debugger Adapter Protocol}\footnote{https://microsoft.github.io/debug-adapter-protocol} (DAP) for modular language workbenches. This endeavor seeks to address significant gaps and challenges developing LSPs and DAPs in the current landscape of language workbenches, particularly in the areas of modularization, composition, and interoperability. Current language workbenches such as Melange~\cite{Degueule15}, MontiCore~\cite{Krahn10}, Spoofax~\cite{Visser10}, and MPS~\cite{Volter11, Voelter12} have made significant strides in supporting modularization, composition, and IDE integration. However, their approaches are often fragmented and lack a standardized method for LSP and DAP generation and modularization.
Neverlang~\cite{Cazzola15c, Cazzola14c}, developed at the \textit{ADAPT-Lab}\footnote{https://di.unimi.it/it/ricerca/risorse-e-luoghi-della-ricerca/laboratori-di-ricerca/adapt-lab} of the Università degli Studi di Milano, being a comprehensive framework for language composition and modularization that supports the development of language product lines~\cite{Cazzola15f, Cazzola21b} (LPLs), is a prime candidate for the implementation of the proposed LSP and DAP. The project will leverage the existing capabilities of Neverlang to develop a universal LSP and DAP that can be used across different programming languages and IDEs. This will enable developers to create external domain-specific languages~\cite{Fowler10} (DSLs) and general-purpose languages (GPLs) more effectively and efficiently, enhancing the overall development experience and productivity.

\hfill \break
The project aims to achieve the following objectives:

\hfill \break
\noindent
\textbf{Aim 1: Improve IDE and LSP Generation}
\hfill \break
\textit{Integrated Development Environment} generation and support for the \textit{Language Server Protocol} are essential for the practical use of domain-specific languages (DSLs). While some language workbenches like Xtext~\cite{Bettini13b} support LSP generation~\cite{Barros22}, many do not, limiting their usability across different editors and IDEs.
\hfill \break
\textbf{Relevance:} By establishing a universal protocol for LSP and DAP, this project aims to bridge the gap, enabling language workbenches to generate IDE support and LSPs more seamlessly. This will ensure that languages developed using these workbenches can be used in any IDE that supports these protocols, enhancing their accessibility and utility.

\hfill \break
\noindent
\textbf{Aim 2: Facilitate LSP and DAP Modularization}
\hfill \break
LSP and DAP modularization are not widely supported by current language workbenches~\cite{Bunder19a}. This feature is crucial for allowing different language components to communicate and function cohesively within an IDE.
\hfill \break
\textbf{Relevance:} Implementing support for LSP and DAP modularization will allow for better integration and interaction of various language features, thereby improving the overall development experience and capability of language workbenches. This aligns with the needs for more sophisticated and integrated language development tools as highlighted in the contemporary research and development literature.

\hfill \break
\noindent
\textbf{Aim 3: Reduce to $\mathcal{O}(\mathcal{L})$ the number of combinations to support $\mathcal{L}$ languages}
\hfill \break
Before the advent of LSP and DAP, developers had to implement language support for each editor separately, having the number of combinations to support $\ell$ languages in $\mathcal{O}(\mathcal{L} \times \mathcal{E})$, where $\mathcal{E}$ is the number of editors.
Currently, the number of combinations to support $\mathcal{L}$ languages is $\mathcal{O}(\mathcal{L} + \mathcal{E})$~\cite{Rodriguez-Echeverria18a}, as the Microsoft LSP and DAP are editor-agnostic. This project aims to reduce the number of combinations to $\mathcal{O}(\mathcal{L})$, by developing a universal LSP and DAP that can be used across different programming languages and IDEs.
\hfill \break
\textbf{Relevance:} Reducing the number of combinations required to support multiple languages will simplify the development process and make it more efficient. This will enable developers to create language support more quickly and effectively, enhancing the overall productivity and usability of language workbenches.

\hfill \break
\noindent
\textbf{Aim 4: Leverage Neverlang for LSP and DAP LPL Development}
\hfill \break
Neverlang's capabilities for language composition and modularization make it an ideal platform for developing a universal LSP and DAP that caters to a variety of language needs. By leveraging Neverlang's LPL development features~\cite{Cazzola20}, the project will establish a reusable core for LSP and DAP functionalities, allowing for the creation of product line variations tailored to specific programming language requirements. This will significantly reduce development time and effort for creating LSPs and DAPs for new languages within the product line.
\hfill \break
\textbf{Relevance:}  Developing a core reusable base for LSP and DAP functionalities through Neverlang's LPL features will streamline the creation of new language support. This fosters a more efficient and scalable approach to LSP and DAP development, aligning perfectly with the core principles of software product lines.


The rapid evolution of software development needs efficient tools for creating and integrating programming languages. \textit{Integrated Development Environments} (IDEs) and \textit{source-code editors} (SCEs) offer vital support features like syntax highlighting, code completion, and debugging, but their development is often complex and labor-intensive. \textbf{Language Server Protocol} (LSP) and \textbf{Debugger Adapter Protocol} (DAP) were introduced to simplify this process by providing a standardized API, decoupling language support implementation from specific editors decreasing the number of combinations from $\mathbf{L} \times \mathbf{E}$ to $\mathbf{L} + \mathbf{E}$. Despite these advancements, the integration of LSP and DAP remains challenging due to fragmented and inconsistent approaches, also caused by it is difficult to reuse IDE implementations across multiple languages. Modern language workbenches have made strides in modularization, composition, and IDE integration. However, their methods for LSP and DAP generation often lack a standardized and cohesive framework, resulting in increased complexity and reduced efficiency. By leveraging techniques like feature-oriented programming and software product lines (SPLs), there is potential to enhance modularity and reusability in language server development. This approach promotes a \textit{bottom-up} methodology where LSP and DAP functionality are encapsulated in feature modules, enabling a more compositional and efficient implementation process. Nowadays, \textit{Xtext}~\cite{Bettini13b} is one of the few language workbenches that support LSP generation~\cite{Barros22}. \textbf{Neverlang}, developed at the \texttt{ADAPT-Lab} of the Universit\'a degli Studi di Milano, is a language workbench focusing on modularity and composition of language features enhancing their reuse.

By extending its capabilities to support a universal LSP and DAP, reusable, language-agnostic feature modules for LSP and DAP would be feasible.. This approach aims to reduce development effort and complexity compared to traditional \textit{top-down} methods. Empirical evidence suggests that a modular framework could significantly improve maintainability, extensibility, and productivity in software~\cite{Sun17}. By applying similar strategies to language support tools, one could achieve similar improvements in language development tools.

Additionally, this project aims to reduce to $\mathbf{L} \times 1$ the number of combinations required to support $\mathbf{L}$ languages. This is important because the number of languages is growing, and the number of editors is limited. By reducing the number of combinations, the effort required to support new languages is reduced, and the quality of language support tools can be improved.

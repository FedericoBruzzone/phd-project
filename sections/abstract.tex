Domain-specific languages (DSLs) and language-oriented programming have become very successful tools in the development of complex software systems. To best suit their purpose, DSLs are very different from one another, yet many of them share commonalities in either their patterns or their implementation. The goal for language designers would be to spot those similarities and to exploit them in order to improve the reuse of preexisting implementations and minimizing the development from scratch. The most common approach for dealing with this task is the use of product line engineering ideas; this introduces the notion of language product line (LPL) to the DSL development. Nowadays, the research has produced several attempts to the creation of tools for variability management. This project will try to take all of them into account but will focus on the tools applying a bottom-up approach to LPL development, i.e. those in which the application engineering phase is performed before the domain engineering phase. Even among these approaches, the support for multi-dimensional variability is scarce. I propose to study a \textit{formal method} to define a \textit{multi-dimensional variability model} that takes each syntactic and sematic role into consideration using a bottom-up approach. As a case-study to translate the formal method into LPL development tools, I propose Neverlang, a framework for modular DSL definition developed by Università degli Studi di Milano, which also already has its own version of syntax-based LPL development tool by means of AiDE.  

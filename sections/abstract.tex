% Domain-specific languages (DSLs) and language-oriented programming have become very successful tools in the development of complex software systems. To best suit their purpose, DSLs are very different from one another, yet many of them share commonalities in either their patterns or their implementation. The goal for language designers would be to spot those similarities and to exploit them in order to improve the reuse of preexisting implementations and minimizing the development from scratch. The most common approach for dealing with this task is the use of product line engineering ideas; this introduces the notion of language product line (LPL) to the DSL development. Nowadays, the research has produced several attempts to the creation of tools for variability management. This project will try to take all of them into account but will focus on the tools applying a bottom-up approach to LPL development, i.e. those in which the application engineering phase is performed before the domain engineering phase. Even among these approaches, the support for multi-dimensional variability is scarce. I propose to study a \textit{formal method} to define a \textit{multi-dimensional variability model} that takes each syntactic and sematic role into consideration using a bottom-up approach. As a case-study to translate the formal method into LPL development tools, I propose Neverlang, a framework for modular DSL definition developed by Università degli Studi di Milano, which also already has its own version of syntax-based LPL development tool by means of AiDE.

The rapid evolution of software development necessitates efficient tools for creating and integrating programming languages. \textit{Integrated Development Environments} (IDEs) and \textit{source-code editors} (SCEs) offer vital support features like syntax highlighting, code completion, and debugging, but their development is often complex and labor-intensive. \textbf{Language Server Protocol} (LSP) and \textbf{Debugger Adapter Protocol} (DAP) were introduced to simplify this process by providing a standardized API, decoupling language support implementation from specific editors. Despite these advancements, the integration of LSP and DAP remains challenging due to fragmented and inconsistent approaches. Modern language workbenches have made strides in modularization, composition, and IDE integration. However, their methods for LSP and DAP generation often lack a standardized and cohesive framework, resulting in increased complexity and reduced efficiency. By leveraging techniques like feature-oriented programming and software product lines (SPLs), there is potential to enhance modularity and reusability in language server development. This approach promotes a \textit{bottom-up} methodology where LSP and DAP functionalities are encapsulated in feature modules, enabling a more compositional and efficient implementation process. Nowadays, \textit{Xtext}~\cite{Bettini13b} is one of the few language workbenches that support LSP generation~\cite{Barros22}. \textbf{Neverlang}, developed at the \texttt{ADAPT-Lab} of the Università degli Studi di Milano, being a framework for language composition and modularization, presents a promising solution. By extending its capabilities to support a universal LSP and DAP, reusable, language-agnostic feature modules can be created. This approach aims to reduce development effort and complexity compared to traditional \textit{top-down} methods. Empirical evidence suggests that a modular framework could significantly improve maintainability, extensibility, and productivity in language support tool development.

